\documentclass[10pt,showpacs,preprintnumbers,footinbib,amsmath,amssymb,aps,prl,twocolumn,groupedaddress,superscriptaddress,showkeys]{revtex4-1}
\usepackage{graphicx}
\usepackage{dcolumn}
\usepackage{bm}
\usepackage[colorlinks=true,urlcolor=blue,citecolor=blue]{hyperref}
\usepackage{color}

%\begin{figure}[hbtp]
%\includegraphics[scale=0.4]{test1.pdf}
%\caption{Exact and numerial solutions for $n=10$ mesh points.} 
%\label{fig:n10points}
%\end{figure}

\begin{document}
\title{Project 1}
\author{Anders Eriksen}
\begin{abstract}
attempting a sleeker version of the general solver of a tridiagonal matrix reducing the general case from ~8n FLOPS to ~4n. n here being the dimensionality of the matrix
\end{abstract}
\maketitle

\section{Introduction}

\section{Theory, algorithms and methods}
We begin with the simple differential equation 
\begin{equation*}
-u''(x) = f(x).
\end{equation*}
which we can discretize as 
\begin{equation*}
   -\frac{v_{i+1}+v_{i-1}-2v_i}{h^2} = f_i  \hspace{0.5cm} \mathrm{for} \hspace{0.1cm} i=1,\dot    s, n,
\end{equation*}
Where the indexes signify preveious, current or next point in our function u. $f_i$ is the disretized $f(x_i) = f_i$. h is the stepsize and is proportional to the number of mesh-points n we have. Our function comes with the conditions that $x \in [0,1]$ and $u(0)=u(1)=0$. An even stepsize then should be $h = \frac{x_{max}- x_{min}}{n} = \frac{1}{n}$. Luckily for our purposes, u has a closed form sollution 
\begin{equation*}
	u(x) = 1-(1-e^{-10})x-e^{-10x}
\end{equation*}
Staring at the discretization for a spell, you can see that the equation can be rewritten as a matrix multiplication of a vector $\vec{u}$: $-\hat{A}\vec{u}=\vec{d}$, with $d_i = h^2\cdot f_i$.  Here, the matrix $\hat{A}$ should be a tri-diagonal matrix of values $a_{11} = b_1 = 2$ along the diagonal and $a_{ii_{\pm 1}} = -1$ to either side. wrting this out index for index makes

 \[
\mathbf{A} = \begin{bmatrix}
                    2& -1& 0 &\dots   & \dots &0 \\
                    -1 & 2 & -1 &0 &\dots &\dots \\
                    0&-1 &2 & -1 & 0 & \dots \\
                    & \dots   & \dots &\dots   &\dots & \dots \\
                    0&\dots   &  &-1 &2& -1 \\
                    0&\dots    &  & 0  &-1 & 2 \\
              \end{bmatrix}
              \begin{bmatrix}
              		u_1 \\
              		u_2 \\
              		\dots \\
              		\dots \\
              		\dots \\
              		u_n
              \end{bmatrix} =
			  \begin{bmatrix}
              		d_1 \\
              		d_2 \\
              		\dots \\
              		\dots \\
              		\dots \\
              		d_n
              \end{bmatrix}              
\]
From this, we can make an upper triangular matrix through the Gaussian row reduction going both forward and backward to solve for $\vec{u}$

\section{Results and discussions}

\section{conclusions}

\section{addendum}
this is a link to the git repo:
https://github.com/andersgeriksen/CompPhys/tree/master/Project1

\end{document}