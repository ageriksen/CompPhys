\documentclass[10pt, twocolumn]{revtex4-1}
\listfiles               %  print all files needed to compile this document

\usepackage{amsmath}
\usepackage{xparse}
\usepackage{graphicx}
\usepackage{dcolumn}
\usepackage{bm}
\usepackage[colorlinks=true,urlcolor=blue,citecolor=blue]{hyperref}
\usepackage{color}
\usepackage{physics}
\usepackage{algorithm2e}
\usepackage{algpseudocode}
\usepackage{natbib}
\usepackage{pgfplots}

\pgfplotsset{compat=1.15}

%\begin{figure}[hbtp]
%\includegraphics[scale=0.4]{test1.pdf}
%\caption{Exact and numerial solutions for $n=10$ mesh points.} 
%\label{fig:n10points}
%\end{figure}

%\begin{tikzpicture}
%    \begin{axis}[
%            title= Earth-Sun system, Forward Euler integration,
%            xlabel={$x$},
%            ylabel={$y$},
%        ]
%        \addplot table {../runresults/earthEuler2body.dat}
%    \end{axis}
%\end{tikzpicture}

\begin{document}

\title{Project 1}
\author{Anders Eriksen}
\begin{abstract}
\end{abstract}
\maketitle


\section{Introduction}

\section{Theory and methods}

The main point of focus in the project is the developement of methods to
solve ordinary differential equations (ODEs) from an object oriented way
mainly through the philosophy of "write once, run repeatedly". The system
we are stydying for this, is a model solar system where the main assumptions
are that the planets being point particles as well as the motion being coplanar. 
We naturally also assume no other forces at play besides gravity. 
\[
F_G=\frac{GM_{\odot}M_{\mathrm{Earth}}}{r^2},
\]
Initially, we also neglect the motion of the sun, as it is overwhelmingly massive.
The origin of the suns orbit is far inside it's radius, which we have assumed to be 
zero. This will be gone into further detail on later, but for a 2 body system of earth 
and sun, we use newtons equations of motion (eom) to get: 
\[
    \frac{d^2\vec{r}}{dt^2}=\frac{\vec{F_{G}}}{M_{\mathrm{Earth}}},
\]


When modeling systems of this size, meters and kilograms are not the best units, as they
are far too small. Instead, we will use astronomical units(AU), sun masses($M_{\odot}$) and years(yr) as our
measurement values. We have then,
\[
    1 AU = 1.5\times 10^{11} \quad 1 M_{\odot} = 2\times 10^{30}
\]
For reference, I include the masses and orbits of relevant planets.\cite{project3pdf}\\
\begin{tabular}{lcc}
\hline
\multicolumn{1}{c}{ Planet } & \multicolumn{1}{c}{ Mass, kg } & \multicolumn{1}{c}{ Distance to sun, AU } \\
\hline
Earth   & $M_{\mathrm{Earth}}=6\cdot 10^{24}$       & 1             \\
Jupiter & $M_{\mathrm{Jupiter}}=1.9\cdot 10^{27}$   & 5.20          \\
Mars    & $M_{\mathrm{Mars}}=6.6\cdot 10^{23}$      & 1.52          \\
Venus   & $M_{\mathrm{Venus}}=4.9\cdot 10^{24}$     & 0.72          \\
Saturn  & $M_{\mathrm{Saturn}}=5.5\cdot 10^{26}$    & 9.54          \\
Mercury & $M_{\mathrm{Mercury}}=3.3\cdot 10^{23}$   & 0.39          \\
Uranus  & $M_{\mathrm{Uranus}}=8.8\cdot 10^{25}$    & 19.19         \\
Neptun  & $M_{\mathrm{Neptun}}=1.03\cdot 10^{26}$   & 30.06         \\
\hline
\end{tabular}\\

Initially, we ensure that our integration methods work, we make a special solution of the earth sun system, implementing
the Forward Euler and velocity Verlet integration methods. Before that, however, we need to discretize our functions. 

If we assume ciruclar motion, then we can rewrite the force as 
\[
F_G= \frac{M_{\mathrm{Earth}}v^2}{r}=\frac{GM_{\odot}M_{\mathrm{Earth}}}{r^2},
\]
and use this to get 
\[
v^2r=GM_{\odot}=4\pi^2\mathrm{AU}^3/\mathrm{yr}^2.
\]
here, $4\pi$ has the dimensions listed. This comes from earth's period being $1 yr$ and it's distance is $1 AU$. 

\begin{algorithm}
    \caption{a Forward Euler integration}
    N $=$ nr. of timesteps \\
    $h = (t_{end} - t_{start})/N $ // steplength \\
    \For{ $i=0$ \KwTo{ $i=N-1$} }{
        \[\vec{ v_{i+1} } += \vec{ a_i } \cdot h\]
        \[\vec{ r_{i+1} } += \vec{ v{i+1} } \cdot h\]
    }
\end{algorithm}

\begin{algorithm}
    \caption{a velocity Verlet integration}
    N $=$ nr. of timesteps \\
    $h = (t_{end} - t_{start})/N $ // steplength \\
    \For{ $i=0$ \KwTo{ $i=N-1$ }}{
        \[\vec{a_i} = \vec{a}(\vec{r_i}i)\]
        \[\vec{r_i{+1}} += h\cdot \vec{v_i} h^2a_i/2\]
        \[\vec{a_{i+1}} = \vec{a}(\vec{r_{i+1}})\]
        \[\vec{v_{i+1}} += h*(a_i + a_{i+1})/2 \]
    }
\end{algorithm}

After ensuring that the algortihms are implemented correctly, I need to generalize the code into an object oriented (OO)
implementation. This should constitute a main file which takes the various objects and binds them together. After looking
through examples and some discussion amongst peers, I have elected to use the methods presented, providing my own "solver" class
and otherwise encorporating code into my experiments. This is in part due to time constraint, but also because using and tweaking
these objects is gonna bring a quicker working knowledge through usage and familiarity. building from scratch can come later.
The main idea in the objects used, is to have one to create a type "planet", which one can then store in another object, the 
"solar system". There is a 3rd one, a "solver" which can find the next position of the planets after a time dt has passed using
the combined gravitational force on the objects from the rest of thes system and the individual positions and velocities. 

As this is an exploration of the numerical methods, we need to explore the results. One way to do this, is to controll for the 
velocity resulting in circular orbit, both through trial and error, as well as through analytical calculation. This would be of interest
comparing Verlet and Euler as well as the time needed to run each, in conjunction with th eamount of FLOPS needed per step in the loops. 
After exploring the 2 methods, further on, we'll use the Verlet method, as this maintains energy conservation. 

Having explored the requirements for circular orbit, we now examine the case for escape velocity. Both through experimentation and analytical analysis. 
As an additional analysis, we will explore how the escape velocity changes as we change the power of r in our expression of force, especially when we lett 
the power approach 3. 

Having verified our equations, we can now complicate our image through adding another body to our problem. We want to add the greatest single contributor
besides the sun, Jupiter. The 3-body problem is far more difficult to process thant the 2-body problem. with 2 bodies, the earthen orbit is stable and
unchanging. Adding in Jupiter, we need to take into account how much of an effect Jupiter has on the earth. Thus, we need the total force as the 
vector sum of the forces from the sun and Jupiter. Now we have pull from different directions over time, and this should affect the orbit of the planet.
The force contribution of Jupiter is on the form
\[
F_{\mathrm{Earth-Jupiter}}=\frac{GM_{\mathrm{Jupiter}}M_{\mathrm{Earth}}}{r_{\mathrm{Earth-Jupiter}}^2},
\]
And we can ensure that the solutions remain stable, by checking that the energy of the system is conserved throuhgt the velocity verlet. 

Once we know that our sollutions are stable for more complicated systems, we can center our system around a center of mass, instead of a stationary sun.
Thus, we now shift the positions to be in regard to the center of mass (COM) of the system. With this, the center of mass will serve as a "stationary" 
point, while all other bodies will revolve around it. To ensure the motion is conserved, we add a velocity to the sun so that the total momentum of the 
system is 0.Finally, we add in the remaining planets of the solar system. The initial positions are all taken from Mathias Vege after some discussion on
Slack.\cite{MathiasPlanetvalues}. 


Now that we have a good working system for our planets, we want to go into some detail on a phenomenon that was crucial to the success of the theory of
relativity, namely the precession of Mercury's perihelion. When all classical effects are accounted for, the observed value of the perihelion precession
is at around 43" per century. This arises from a small perturbation of the $1/r^2$ dependency of the gravitationsl force which results in the effective 
precession of the elipse. The correction to the force becomes 
\[
F_G = \frac{GM_\mathrm{Sun}M_\mathrm{Mercury}}{r^2}\left[1 + \frac{3l^2}{r^2c^2}\right]
\]
Where $l=|\vec{r}\times\vec{v}|$ is the magnitude of Mercury's orbital angular momentum per unit mass, with c as the speed of light in vacuum. We trac the
angle of perihelion $\theta_P$ with 
\[
\tan \theta_\mathrm{p} = \frac{y_\mathrm{p}}{x_\mathrm{p}}
\]
or the ratio of Mercury's x and y position at perihelion. Using the perihelion speed of Mercury as $12.44\,\mathrm{AU}/\mathrm{yr}$ with a distance to the 
sun of $0.3075\,\mathrm{AU}$. As a controll of the reoslution, we can compare with the precession of the perihelion without the correction. 


\section{Results and discusion}

For the initial incarnation of the solution, just a simple run of either Verlet or Euler with the Earth-Sun system, we can see a change in the distance
for the Euler solution, but not so for the Verlet solver. This indicates that the energy really is conserved in the integration. for 2 body problem, the
question of a circular orbit is fairly simple, merely following the sentripetal acceleration $a = \frac{v^2}{r}$ which gives us a velocity of 
$v = \frac{2\pi }{r}$

\begin{figure}[hbtp]
    \includegraphics[scale=0.4]{2bodyEuler.png}
    \caption{Forward Euler integration of the 2 body system of a stationary sun and earth. 
        There is a clear drift as time goes by, which empathizes the drif in energy of the 
        system with the Euler, since the integrations are done in a conservative potential
        field.} 
    \label{}
\end{figure}

The Forward Eulre integration has a clear drift over time as seen in figure 1, stemming from a failiure to conserve the energy of the system. 
The velocity verlet, however maintains the energy, see figure 2, as well as having a numerical error proportional to steplength $h^2$, as 
opposed to Euler's $h$. We can also see that the resulting orbit is circular, as we should expect from the initial velocity of the simpe system. 


\begin{figure}[hbtp]
    \includegraphics[scale=0.4]{2bodyVerlet.png}
    \caption{Velocity Verlet integration of the 2 body system of a stationary sun and earth.
        This integration is run with as many steps as figure 1, which shows how much more robust
        the Verlet solution is compared to the Forward Euler. We can also see from the axes that 
        despite the seeming eliptical orbit, this has more to do with improper scaling. The radius
        is 1 AU in all directions. } 
    \label{}
\end{figure}

When calculating escape velociy, I used a distance $r= 1AU$ and used the fact that the energy would need to be either $0$
or greater if the planet was to escape. Therefore, I sat $E_k = E_p$ and found a velocity of $v = \sqrt{\frac{8\pi^2}{r}}$,
the initial distance meant $v = 2\sqrt{2}\pi$. This is ilustrated in figure 3, where we see that the motion is less and 
less affected by the star with distance. As this is the exact escape velocity, it will not disappear until the distance is at 
infinity. Currently, the velocity goes according to $v \propto \frac{1}{\sqrt{r}}$, but if we were to increase the density
dropoff of the gravitational force according to the cube of the distance, then the escape velocity would approach 
$v \propto \frac{1}{r}$.
\begin{figure}[hbtp]
    \includegraphics[scale=0.4]{escapevelocity.png}
    \caption{this is the earths "orbit" around the sun with an escape velocity so $E_k = E_p \rightarrow E=0$.
        For a system at a distance of $1 AU$, escape velocity is, as stated in the title. Measurements show that 
        just below leads to a stable orbit with a change from $2 \rightarrow 1.9$ in front of the square root.} 
    \label{}
\end{figure}

At this point, we begin to leave the realm of analytical solutions explored by Newton and his contemporaries. Here, we see that 
the inclusion of Jupiter squishes the circular orbit of the Earth into an elipse. This is also a challenge to pick good values for 
initial conditions. Therefore, I use information on planet positions and velocities\cite{MathiasPlanetvalues} as mentioned above. 
\begin{figure}[hbtp]
    \includegraphics[scale=0.4]{jupiterXmass4.png}
    \caption{The orbits of Earth and Jupiter around a stationary sun for varying masses of jupiter. We can see 
    as we increase the mass from Jupoter's regular mass(top left) by $10$(top right) and $1 000$(botom row) 
    respectively, we get increasingly rampant orbits, with the lower spiraling out completely. We can see that as
    Jupiter's mass approaches that of the sun, it destabilizes the system. }
    \label{}
\end{figure}
Varying Jupiter's mass, we can see that it provides increasingly greater distractions from the sun with regards to the Earth untill
Jupiter is as massive as the sun. At this point, the system breaks down and spirals outwards. What we can notice throughout these, is
that the momentum of the system remains the same to a relative error $\epsilon < 1e-5$, which means that the Verlet method still holds 
even under these strange developements. 

\begin{figure}[hbtp]
    \includegraphics[scale=0.4]{stationarySun3Body.png}
    \caption{An enlarged version of the first plot in figure 4. We can see, that the previously circular orbit of
        the earth is pushed to be eliptical. We also see some slight axial tilt in the axis. the Verlet method retains
        angular momentum to a great deal, and this should be close to the effect of the planet, rather than iancuracies 
        in the numerical method.}
    \label{}
\end{figure}


As we have now more or less verified our numericals for various systems and number of planets, we can begin to add in both the motion 
of the sun as well as the remaining planets(excluding Pluto which is currently not within this deifnition) in figures 6 and 7. We can observe a gap between
the orbits that seem to almost increase exponentially with the distance form the sun. We can also see that the 5 decades of time developement 
does not seem to be enough to complete the orbits of the furthest. planets. This makes sense, as the gravitational field reduces with 
$\frac{1}{r}$, and so the resulting velocities that remain bound to the system have to be significantly lower, as the kinetic energy is 
proportional to the square of the velocities. we can also see, that the initial assumption of coplanar motion is off, but not by terribly 
much over all, with some tilt in either direction. 

\begin{figure}[hbtp]
    \includegraphics[scale=0.4]{fullSystemSameAngle.png}
    \caption{The full solar system (excluding Pluto). The data is gathered over 5 decades, which is not quite enough 
        time for the outermost planets to complete an orbit, but we see that Saturn and closer planets have seemingly 
        closed orbits. There is a distinct distance between the outer planets Neptune, Uranus, Saturn and, to a degree, 
        Jupiter and the inner planets beginning with mars. This inner part has far tighter orbits in comparison.} 
    \label{}
\end{figure}


\begin{figure}[hbtp]
    \includegraphics[scale=0.4]{fullSystemNewAngle.png}
    \caption{A shifted angle of the same plot as figure 6, in an attempt to better showcase the inner "nucleus". 
        We can see a progressively closer orbits as we approach the sun, almost seeming to follow an inverse proportionality 
        to the distance from the sun. } 
    \label{}
\end{figure}

A final consideration, is the accuracy of our potential calculations. We assumed the system followed the newtonian gravity, which seems 
to be fairly accurate on the whole. But there are corrections that can be made with regards to the theories of relativity. In particular, 
an important test of Einsteins theories was the perihelion precession of Mercury. The perihelion precession was measured to be $43''$ per century.
Newtonian gravity predicts several oreders of magnitude less in this timeframe. To examine this, I ran a system of pure Sun-Mercury and set Mercury's 
initial values as those in the perihelion. Letting the simulation run for a century with a sampling of $1e8-1e9$ points, I found a precession of 
$42.8''$ for the relativistically corrected equation presented in the theory above. While the newtonian system predicts a precession of $0.02''$.
this shows that the relativistic model is about $1 000$ more accurate at predicting systems of this magnitude than newton's. 

\section{conclusion}
As we built up our program and system, we saw that the velocity Verlet method vastly outperformed the Forward Euler, maintaining a good approximation
to the plante's path with around a few hundred integration points a day during our revolutions around the star and maintaining a near constant angular
momentum through the years in out isolated system. \\

As we added complexity to our system, we also developed a program with objects that could 
compartmentalize the required tasks and feed the finished results into eachother, resulting in a program that could near seemlessly add planets and 
calculate their positions and velocities over time without more than an added line. This object orientation mainly revolced around 4 objects. A "celestial 
body" class, which took in the planets and assigned them positions, velocities and other physical observables which could then be accessed down the line. 
A "system" class which stored all the planets and could work out the forces inbetween each of them and a "Solver" class, which could solve the equations of 
motion for the planets in the system a step forwards in time. This way of handling things allowed for far less repetition of definitions and helped insulate
mistakes.\\

While the systems we set up worked well, there are a few things that I did not have the time to fully automate, such as the force used, or the calls to various 
amounts of planets. Ideally, I envision a system in which I can give a filename and the program will implement a planet with name, position, vellocity and 
other attributes to create the system I wish to study. further, I would like to customize the span of the integration in both length and timestep or number of
steps. The main issue I encountered on this front, was in innterpreting the input. I believe a sollution here could be a textfile as input, with relevant info
such as data files, integration length and similar, while the system itself has perhaps another object that can interpret input from file and put the results
where they ought to go. Finally, I should have specified a folder to write the results into. \\
Another solution could be a python script to handle inputs and outputs, with plots as necessary. This would be a more involved project than what time allows 
for this time around. 

\section{appendix}

\bibliography{\string~/Documents/bibliography/Bibliography}
\end{document}
