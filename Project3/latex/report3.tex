\documentclass[10pt, twocolumn]{revtex4-1}
\listfiles               %  print all files needed to compile this document

\usepackage{amsmath}
\usepackage{xparse}
\usepackage{graphicx}
\usepackage{dcolumn}
\usepackage{bm}
\usepackage[colorlinks=true,urlcolor=blue,citecolor=blue]{hyperref}
\usepackage{color}
\usepackage{physics}
\usepackage{algorithm2e}
\usepackage{algpseudocode}
\usepackage{natbib}

\setcitestyle{numbers}

%\begin{figure}[hbtp]
%\includegraphics[scale=0.4]{test1.pdf}
%\caption{Exact and numerial solutions for $n=10$ mesh points.} 
%\label{fig:n10points}
%\end{figure}


\begin{document}

\title{Project 1}
\author{Anders Eriksen}
\maketitle

%\begin{abstract}
%\end{abstract}

\section{Introduction}

\section{Theory and methods}

The main point of focus in the project is the developement of methods to
solve ordinary differential equations (ODEs) from an object oriented way
mainly through the philosophy of "write once, run repeatedly". The system
we are stydying for this, is a model solar system where the main assumptions
are that the planets being point particles as well as the motion being coplanar. 
We naturally also assume no other forces at play besides gravity. 
\[
F_G=\frac{GM_{\odot}M_{\mathrm{Earth}}}{r^2},
\]
Initially, we also neglect the motion of the sun, as it is overwhelmingly massive.
The origin of the suns orbit is far inside it's radius, which we have assumed to be 
zero. This will be gone into further detail on later, but for a 2 body system of earth 
and sun, we use newtons equations of motion (eom) to get: 
\[
    \frac{d^2\vec{r}}{dt^2}=\frac{\vec{F_{G}}}{M_{\mathrm{Earth}}},
\]


When modeling systems of this size, meters and kilograms are not the best units, as they
are far too small. Instead, we will use astronomical units(AU), sun masses($M_{\odot}$) and years(yr) as our
measurement values. We have then,
\[
    1 AU = 1.5\times 10^{11} \quad 1 M_{\odot} = 2\times 10^{30}
\]
For reference, I include the masses and orbits of relevant planets.\cite{project3pdf}
\begin{quote}
\begin{tabular}{ccc}
\hline
\multicolumn{1}{c}{ Planet } & \multicolumn{1}{c}{ Mass in kg } & \multicolumn{1}{c}{ Distance to  sun in AU } \\
\hline
Earth   & $M_{\mathrm{Earth}}=6\times 10^{24}$ kg     & 1AU                    \\
Jupiter & $M_{\mathrm{Jupiter}}=1.9\times 10^{27}$ kg & 5.20 AU                \\
Mars    & $M_{\mathrm{Mars}}=6.6\times 10^{23}$ kg    & 1.52 AU                \\
Venus   & $M_{\mathrm{Venus}}=4.9\times 10^{24}$ kg   & 0.72 AU                \\
Saturn  & $M_{\mathrm{Saturn}}=5.5\times 10^{26}$ kg  & 9.54 AU                \\
Mercury & $M_{\mathrm{Mercury}}=3.3\times 10^{23}$ kg & 0.39 AU                \\
Uranus  & $M_{\mathrm{Uranus}}=8.8\times 10^{25}$ kg  & 19.19 AU               \\
Neptun  & $M_{\mathrm{Neptun}}=1.03\times 10^{26}$ kg & 30.06 AU               \\
Pluto   & $M_{\mathrm{Pluto}}=1.31\times 10^{22}$ kg  & 39.53 AU               \\
\hline
\end{tabular}
\end{quote}

Initially, we ensure that our integration methods work, we make a special solution of the earth sun system, implementing
the Forward Euler and velocity Verlet integration methods. Before that, however, we need to discretize our functions. 

If we assume ciruclar motion, then we can rewrite the force as 
\[
F_G= \frac{M_{\mathrm{Earth}}v^2}{r}=\frac{GM_{\odot}M_{\mathrm{Earth}}}{r^2},
\]
and use this to get 
\[
v^2r=GM_{\odot}=4\pi^2\mathrm{AU}^3/\mathrm{yr}^2.
\]
here, $4\pi$ has the dimensions listed. This comes from earth's period being $1 yr$ and it's distance is $1 AU$. 

\begin{algorithm}
    \caption{a Forward Euler integration}
\end{algorithm}

\begin{algorithm}
    \caption{a velocity Verlet integration}
\end{algorithm}

After ensuring that the algortihms are implemented correctly, I need to generalize the code into an object oriented (OO)
implementation. This should constitute a main file which takes the various objects and binds them together. After looking
through examples and some discussion amongst peers, I have elected to use the methods presented, providing my own "solver" class
and otherwise encorporating code into my experiments. This is in part due to time constraint, but also because using and tweaking
these objects is gonna bring a quicker working knowledge through usage and familiarity. building from scratch can come later.
The main idea in the objects used, is to have one to create a type "planet", which one can then store in another object, the 
"solar system". There is a 3rd one, a "solver" which can find the next position of the planets after a time dt has passed using
the combined gravitational force on the objects from the rest of thes system and the individual positions and velocities. 

As this is an exploration of the numerical methods, we need to explore the results. One way to do this, is to controll for the 
velocity resulting in circular orbit, both through trial and error, as well as through analytical calculation. This would be of interest
comparing Verlet and Euler as well as the time needed to run each, in conjunction with th eamount of FLOPS needed per step in the loops. 
After exploring the 2 methods, further on, we'll use the Verlet method, as this maintains energy conservation. 

Having explored the requirements for circular orbit, we now examine the case for escape velocity. Both through experimentation and analytical analysis. 
As an additional analysis, we will explore how the escape velocity changes as we change the power of r in our expression of force, especially when we lett 
the power approach 3. 

Having verified our equations, we can now complicate our image through adding another body to our problem. We want to add the greatest single contributor
besides the sun, Jupiter. The 3-body problem is far more difficult to process thant the 2-body problem. with 2 bodies, the earthen orbit is stable and
unchanging. Adding in Jupiter, we need to take into account how much of an effect Jupiter has on the earth. Thus, we need the total force as the 
vector sum of the forces from the sun and Jupiter. Now we have pull from different directions over time, and this should affect the orbit of the planet.
The force contribution of Jupiter is on the form
\[
F_{\mathrm{Earth-Jupiter}}=\frac{GM_{\mathrm{Jupiter}}M_{\mathrm{Earth}}}{r_{\mathrm{Earth-Jupiter}}^2},
\]
And we can ensure that the solutions remain stable, by checking that the energy of the system is conserved throuhgt the velocity verlet. 

Once we know that our sollutions are stable for more complicated systems, we can center our system around a center of mass, instead of a stationary sun.
Thus, we now shift the positions to be in regard to the center of mass (COM) of the system. With this, the center of mass will serve as a "stationary" 
point, while all other bodies will revolve around it. To ensure the motion is conserved, we add a velocity to the sun so that the total momentum of the 
system is 0.Finally, we add in the remaining planets of the solar system. The initial positions are all taken from the NASA link 
provided\cite{NASAPlanetvalues}. 


Now that we have a good working system for our planets, we want to go into some detail on a phenomenon that was crucial to the success of the theory of
relativity, namely the precession of Mercury's perihelion. When all classical effects are accounted for, the observed value of the perihelion precession
is at around 43" per century. This arises from a small perturbation of the $1/r^2$ dependency of the gravitationsl force which results in the effective 
precession of the elipse. The correction to the force becomes 
\[
F_G = \frac{GM_\mathrm{Sun}M_\mathrm{Mercury}}{r^2}\left[1 + \frac{3l^2}{r^2c^2}\right]
\]
Where $l=|\vec{r}\times\vec{v}|$ is the magnitude of Mercury's orbital angular momentum per unit mass, with c as the speed of light in vacuum. We trac the
angle of perihelion $\theta_P$ with 
\[
\tan \theta_\mathrm{p} = \frac{y_\mathrm{p}}{x_\mathrm{p}}
\]
or the ratio of Mercury's x and y position at perihelion. Using the perihelion speed of Mercury as $12.44\,\mathrm{AU}/\mathrm{yr}$ with a distance to the 
sun of $0.3075\,\mathrm{AU}$. As a controll of the reoslution, we can compare with the precession of the perihelion without the correction. 


\section{Results and discusion}

\section{conclusion}

\section{appendix}

\bibliography{\string~/Documents/bibliography/Bibliography}
\end{document}
