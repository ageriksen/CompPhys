\documentclass[10pt]{revtex4-1}
\listfiles               %  print all files needed to compile this document

\usepackage{amsmath}
\usepackage{xparse}
\usepackage{graphicx}
\usepackage{dcolumn}
\usepackage{bm}
\usepackage[colorlinks=true,urlcolor=blue,citecolor=blue]{hyperref}
\usepackage{color}
\usepackage{physics}
\usepackage{algorithm2e}
\usepackage{algpseudocode}
\usepackage{pgfplots}
\usepackage{pgfplotstable, booktabs, mathpazo}
\usepackage{natbib}

\pgfplotsset{compat=1.15}

\pgfplotstableset{
    every head row/.style={before row=\toprule \hline ,after row=\hline\hline \midrule},
    every last row/.style={after row=\hline \bottomrule},
    every first column/.style={
        column type/.add={|}{}
        },
    every last column/.style={
        column type/.add={}{|}
        },
}

%\begin{figure}[hbtp]
%\includegraphics[scale=0.4]{.pdf}
%\caption{}
%\label{fig:}
%\end{figure}

%\begin{tikzpicture}
%    \begin{axis}[
%            title= Earth-Sun system, Forward Euler integration,
%            xlabel={$x$},
%            ylabel={$y$},
%        ]
%        \addplot table {../runresults/earthEuler2body.dat}
%    \end{axis}
%\end{tikzpicture}

\begin{document}
\title{%
    Project 5\\
    \large
    Quantum Monte Carlo of Confined Electrons}
\author{Anders Eriksen}
\begin{abstract}
\end{abstract}
\maketitle

\section{Introduction}
Aim: VMC to evaluate gs.\  energy, relative distance between 2 electrons as well as expectationvalues of kinetic and potenial energy of quantum dots
with $N=2$ electrons in 3 dimensions.

A stochastic approach to problem in project 2. Though most studies are done in the 2 dimensional case, this will be kept in 3D to more easiliy
compare with the project mentioned.

In section i, I go through ... and ... and finally .... And in section ii, I provide and discuss my findings. Finally, in section iii, I concretize
the results and discuss possible future paths as well as points at which this project could have been made better.

\section{Theory and Methods}
Quantum dots are a lively research area for condensed matter physics and material science with wide reaching applications from quantum computing
to solar cells.

Main aim is to study these systems with a focus on understanding the repulsive forces between the electrons. The advantage of the VMC approach is the
ability to use cartesian coordinates over polar. %% Relevant background material e.g. chapter 14 of the lecture notes.

Our system is of electrons confined in a pure 3D isotropic harmonic oscillator potential with an idealized total Hamiltonian
\begin{align}
    \hat{H} = \sum_{i = 1}^{N} \qty( - \frac{1}{2} \nabla_{i}^{2} + \frac{1}{2} \omega^{2} r_{i}^{2} ) + \sum_{i < j} \frac{1}{r_{i j}}\label{eq:full}
\end{align}
having set the natural units $\hbar = c = e = m_{e} = 1$, as well as all put all energies in atomic units $a.u$.

Thus all results will be scaled. The first sum
is the standard harmonic oscillator, whilst the latter part of the Hamiltonian is the repulsive forces between the electrons.
\begin{align}
    \hat{H}^0 &= \sum_{i = 1}^{N} \qty( - \frac{1}{2} \nabla_{i}^{2} + \frac{1}{2} \omega^{2} r_{i}^{2} )\label{eq:unperturbed} \\
    \hat{H}'  &= \sum_{i < j} \frac{1}{r_{i j}} \label{eq:perturbed}
\end{align}
The distance
$r_{ij} = \sqrt{\vec{r_1} - \vec{r_2}}$ is between the 2 particles, each position given as $r_i = \sqrt{ x_i^2 + y_i^2 + z_i^2 }$.

The first step will be the unperturbed harmonic oscillator Hamiltonian mentioned above (\ref{eq:unperturbed}). Setting $\hbar \omega = 1$ gives the energy
$3 a.u$. This will serve as a benchmark during developement of the code.

In 3 dimensions, the wave equation is
\begin{align}
    \psi_{\vec{n}} \qty(\vec{r}) = A H_{n_x}(\sqrt{\omega} \, x )\, H_{n_y}(\sqrt{\omega} \, y )\, H_{n_z}(\sqrt{\omega} \, z )
    e^{-\omega \, r^2 / 2}
\end{align}
The $H_{n_i}(\omega\, i )$ are the Hermite polynomials and A is a normalization constant.\ for the ground state, and assuming opposing spins, the
energy becomes $\epsilon_{\vec{n}} = \frac{3}{2}$ per electron. This brings the total energy up to $3 a.u$ as stated previuosly.
The unperturbed ground state eigenfunction for the 2 electron system then becomes
\begin{align}
    \vec{\Psi}(\vec{r_1}, \vec{r_2} ) = C e^{-\frac{1}{2}\omega \, ( r_1^2 + r_2^2 ) }
\end{align}
With C as a normalization constant.

Since the electrons are fermions, they obey the Pauli exclusion principle. This means that the ground state is
non-degenerate with each having opposing spin. This could also be likened to the singlet state in the nlm style Dirac notation. The total spin of this
system must therefore be 0.

Our first two trial wave functions will be
\begin{align}
    \Psi_{T_1} \qty( \vec{r}_{1}, \vec{r}_{2} ) &= C e^{-\frac{1}{2}\alpha \omega \qty(r_{1}^{2} + r_{2}^{2})}\label{eq:trial1}\\
    \Psi_{T_2} \qty( \vec{r}_{1}, \vec{r}_{2} ) &= C e^{-\frac{1}{2}\alpha \omega \qty(r_{1}^{2} + r_{2}^{2})} e^{\frac{r_{12}}{2\qty(1 + \beta r_{12})}}
    \label{eq:trial2}
\end{align}
With $\alpha$ and $\beta$ as variational parameters.

The energy at a given distance, $r_1$, $r1_2$ is our local energy
\begin{align}
    E_{L_1} &= \frac{H \Psi_{T_1} \qty(\vec{r}_1, \vec{r}_2, \alpha )}{\Psi_{T_1} \qty(\vec{r}_1, \vec{r}_2, \alpha )}\nonumber \\
        &= C^{-1} e^{\frac{1}{2}\alpha \omega \qty(r_{1}^{2} + r_{2}^{2})} \frac{1}{2}\qty(-\nabla_1^2 -\nabla_2^2 + \omega^2(r_1^2 + r_2^2))
          C e^{-\frac{1}{2} \alpha \omega \qty(r_1^2 + r_2^2)} \nonumber \\
        &= e^{\frac{1}{2}\alpha \omega \qty(r_{1}^{2} + r_{2}^{2})} \frac{1}{2}\qty(-\alpha^2 \omega^2 (r_1^2 + r_2^2) + \omega^2 (r_1^2 + r_2^2))
          e^{-\frac{1}{2} \alpha \omega \qty(r_1^2 + r_2^2)} \nonumber \\
        &= \frac{1}{2}\omega^2 (r_1^2 + r_2^2)\qty(1 -\alpha^2 ) + 3\alpha \omega \label{eq:naiveLocalEnergy}
\end{align}
Where we use $\nabla_i^2 = \pdv{x_i}^2 + \pdv{y_i}^2 + \pdv{z_i}^2$ and, as an example,
\begin{align}
    \pdv[2]{x_1} \Psi{T_1} &= \pdv{x_1}\qty( -\alpha \omega x_1 e^{-\frac{1}{2}(r_1^2 + r_2^2)} )\nonumber\\
                                &= -\alpha\omega + (\alpha\omega x_1^2)e^{-\frac{1}{2}(r_1^2 + r_2^2)}
\end{align}
Summing upp all 3 dimensions, gives us the answer in (\ref{eq:naiveLocalEnergy}). As the repulsive Coloumb interaction is purely dependent on positions
and works between particles, adding it to our simple 2-particle state results in
\begin{align}
    E_{L_} = \frac{1}{2}\omega^2 (r_1^2 + r_2^2)\qty(1 -\alpha^2 ) + 3\alpha \omega \label{eq:firstLocalEnergy}
\end{align}

\section{Results}


\begin{table}[h!tb]
    \centering
    \caption{Table of first run}
    \pgfplotstabletypeset[sci]{../data/TwoParticlesNonInteractiveWF.dat}
\end{table}

\section{Conclusion}

\section{Appendix}

\end{document}
