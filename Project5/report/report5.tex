\documentclass[10pt]{revtex4-1}
\listfiles               %  print all files needed to compile this document

\usepackage{amsmath}
\usepackage{xparse}
\usepackage{graphicx}
\usepackage{dcolumn}
\usepackage{bm}
\usepackage[colorlinks=true,urlcolor=blue,citecolor=blue]{hyperref}
\usepackage{color}
\usepackage{physics}
\usepackage{algorithm2e}
\usepackage{algpseudocode}
\usepackage{pgfplots}
\usepackage{pgfplotstable, booktabs, mathpazo}
\usepackage{natbib}

\pgfplotsset{compat=1.15}

%\pgfplotstableset{
%    every head row/.style={before row=\toprule \hline ,after row=\hline\hline \midrule},
%    every last row/.style={after row=\hline \bottomrule},
%    every first column/.style={
%        column type/.add={|}{}
%        },
%    every last column/.style={
%        column type/.add={}{|}
%        },
%}
\pgfplotstableset{
    every head row/.style={before row=\toprule \hline ,after row=\hline\hline \midrule},
    every last row/.style={after row=\hline \bottomrule}
}

%\begin{figure}[hbtp]
%\includegraphics[scale=0.4]{.pdf}
%\caption{}
%\label{fig:}
%\end{figure}

%\begin{tikzpicture}
%    \begin{axis}[
%            title= Earth-Sun system, Forward Euler integration,
%            xlabel={$x$},
%            ylabel={$y$},
%        ]
%        \addplot table {../runresults/earthEuler2body.dat}
%    \end{axis}
%\end{tikzpicture}

\begin{document}
\title{%
    Project 5\\
    \large
    Quantum Monte Carlo of Confined Electrons}
\author{Anders Eriksen}
\begin{abstract}
\end{abstract}
\maketitle

\section{Introduction}
Aim: VMC to evaluate gs.\  energy, relative distance between 2 electrons as well as expectationvalues of kinetic and potenial energy of quantum dots
with $N=2$ electrons in 3 dimensions.

A stochastic approach to problem in project 2. Though most studies are done in the 2 dimensional case, this will be kept in 3D to more easiliy
compare with the project mentioned.

In section i, I go through ... and ... and finally .... And in section ii, I provide and discuss my findings. Finally, in section iii, I concretize
the results and discuss possible future paths as well as points at which this project could have been made better.

\section{Theory and Methods}
Quantum dots are a lively research area for condensed matter physics and material science with wide reaching applications from quantum computing
to solar cells.

Main aim is to study these systems with a focus on understanding the repulsive forces between the electrons. The advantage of the VMC approach is the
ability to use cartesian coordinates over polar. %% Relevant background material e.g. chapter 14 of the lecture notes.

Our system is of electrons confined in a pure 3D isotropic harmonic oscillator potential with an idealized total Hamiltonian
\begin{align}
    \hat{H} = \sum_{i = 1}^{N} \qty( - \frac{1}{2} \nabla_{i}^{2} + \frac{1}{2} \omega^{2} r_{i}^{2} ) + \sum_{i < j} \frac{1}{r_{i j}}\label{eq:full}
\end{align}
having set the natural units $\hbar = c = e = m_{e} = 1$, as well as all put all energies in atomic units $a.u$.

Thus all results will be scaled. The first sum
is the standard harmonic oscillator, whilst the latter part of the Hamiltonian is the repulsive forces between the electrons.
\begin{align}
    \hat{H}^0 &= \sum_{i = 1}^{N} \qty( - \frac{1}{2} \nabla_{i}^{2} + \frac{1}{2} \omega^{2} r_{i}^{2} )\label{eq:unperturbed} \\
    \hat{H}'  &= \sum_{i < j} \frac{1}{r_{i j}} \label{eq:perturbed}
\end{align}
The distance
$r_{ij} = \sqrt{\vec{r_1} - \vec{r_2}}$ is between the 2 particles, each position given as $r_i = \sqrt{ x_i^2 + y_i^2 + z_i^2 }$.

The first step will be the unperturbed harmonic oscillator Hamiltonian mentioned above (\ref{eq:unperturbed}). Setting $\hbar \omega = 1$ gives the energy
$3 a.u$. This will serve as a benchmark during developement of the code.

In 3 dimensions, the wave equation is
\begin{align}
    \psi_{\vec{n}} \qty(\vec{r}) = A H_{n_x}(\sqrt{\omega} \, x )\, H_{n_y}(\sqrt{\omega} \, y )\, H_{n_z}(\sqrt{\omega} \, z )
    e^{-\omega \, r^2 / 2}
\end{align}
The $H_{n_i}(\omega\, i )$ are the Hermite polynomials and A is a normalization constant.\ for the ground state, and assuming opposing spins, the
energy becomes $\epsilon_{\vec{n}} = \frac{3}{2}$ per electron. This brings the total energy up to $3 a.u$ as stated previuosly.
The unperturbed ground state eigenfunction for the 2 electron system then becomes
\begin{align}
    \vec{\Psi}(\vec{r_1}, \vec{r_2} ) = C e^{-\frac{1}{2}\omega \, ( r_1^2 + r_2^2 ) }
\end{align}
With C as a normalization constant.

Since the electrons are fermions, they obey the Pauli exclusion principle. This means that the ground state is
non-degenerate with each having opposing spin. This could also be likened to the singlet state in the nlm style Dirac notation. The total spin of this
system must therefore be 0.

Our first two trial wave functions will be
\begin{align}
    \Psi_{T_1} \qty( \vec{r}_{1}, \vec{r}_{2} ) &= C e^{-\frac{1}{2}\alpha \omega \qty(r_{1}^{2} + r_{2}^{2})}\label{eq:trial1}\\
    \Psi_{T_2} \qty( \vec{r}_{1}, \vec{r}_{2} ) &= C e^{-\frac{1}{2}\alpha \omega \qty(r_{1}^{2} + r_{2}^{2})} e^{\frac{r_{12}}{2\qty(1 + \beta r_{12})}}
    \label{eq:trial2}
\end{align}
With $\alpha$ and $\beta$ as variational parameters.

The energy at a given distance, $r_1$, $r1_2$ is our local energy
\begin{align}
    E_{L_1} &= \frac{H \Psi_{T_1} \qty(\vec{r}_1, \vec{r}_2, \alpha )}{\Psi_{T_1} \qty(\vec{r}_1, \vec{r}_2, \alpha )}\nonumber \\
        &= \frac{1}{2}\omega^2 (r_1^2 + r_2^2)\qty(1 -\alpha^2 ) + 3\alpha \omega \label{eq:naiveLocalEnergy}
\end{align}
The road there is found in (\ref{eq:naiveLocalEnergyExpanded}). As this has a well defined and simple energy function, it is a good target for benchmarking.
As the repulsive Coloumb interaction is purely dependent on positions
and works between particles, adding it to our simple 2-particle state results in
\begin{align}
    E_{L_1} = \frac{1}{2} \omega^2 \qty(r_1^2 + r_2^2) \qty( 1 -\alpha^2 ) + 3\alpha \omega + \frac{1}{ r_{12} } \label{eq:firstLocalEnergy}
\end{align}
Which is eschewed somewhat by the electron interaction. We can approach a best estimate by varying $\alpha$. The results of this can also be used to compare
the second trial function once total energy has been found for both.

The second trial wavefunction, (\ref{eq:trial2}) gives a few extra products, but thanks to the product rule, they separate into
\begin{align}
    E_{L_2} = E_{L_1} + \frac{1}{ 2 \qty( 1 + \beta r_{12} )^{2} }
    \qty[ \alpha \omega r_{12} - \frac{ 1 }{ 2 \qty( 1 + \beta r_{2} )^{2} } - \frac{ 2 }{ r_{12} } + \frac{ 2 \beta }{ 1 + \beta r_{12} } ]
\end{align}

And has an exact energy for $\omega = 1$ of $3.558a.u$. this can be compared to the eigenvalue approach to the quantum dot\cite{project2pdf}.

The chosen integrator over these quantum states is the variational Monte Carlo method. We thus want to apply the variational principle for approaching
the ground state of a pair of electrons in a harmonic oscillator potential, assuming a singlet spin state. this means the spatial equations need to be
symetric to accomodate the Pauli exclusion principle. % REFERENCE
this fits well with the chosen trial wave functions, as they are symetric under spatial parity. For the variational method, we need to find the trial
energy,
\begin{align}
    \ev{H} &= \frac{ \ev{H}{\Psi_T} }{ \braket{\Psi_T} } \qq{see appendix} (\ref{App:trialEnergy}) \nonumber \\
        &= \int\dd\vec{r} E_L\qty(\vec{r}) P\qty(\vec{r}) \qq{which discretizes to } \nonumber \\
        &= \frac{1}{N_{MC}}\sum_i^{N_{MC}} P(\vec{r}) \, E_L(\vec{r}) \qq{with possibly extra variational parameters.}
\end{align}

For the MC sample, there needs to be an acceptance rule as well as a method for samping each step. The metropilis sampling equation is a decent way of deciding,
considering the possibility of finding ratios between probabilities from the wavefunction. As a sampling rule, the basic uniform distribution works.
Though this means discarding values outside the acceptance, it gives an acceptable result.

%----------------------------------------------------------------------------------------------------------------------------------------------
%   ALGORITHMS. SAMPLE FROM MC SIMULATINO OF LATTICE MAGNET
%----------------------------------------------------------------------------------------------------------------------------------------------
The stepLength is found by running through a shorter sequence, minimizing the variance during runtime and stoping once the acceptance rate is
acceptably close to $50\%$.
%\begin{algorithm}
%
%\end{algorithm}

\begin{algorithm}
    \For{ each particle }{
        \For{ each dimension }{
            1: Take a random step of length $rnd\vdot \delta$, $rnd \in [-1,1]$ \\
            2: check the energetic gain in taking the step.\\
            3: compare probability ratio between the 2 positions to a stochastic number $ a \in {0.0, 1.0}$\\
            \If{ $a <= \Pr{\Delta E}$ }{
                apply change\\
                update positions and count
                as an accepted configuration
                }
            \Else{ Keep old positions }

        }
    }
    | Calculate forces.
    \caption{The algorithm for 1 MC cycle. }
\end{algorithm}
%----------------------------------------------------------------------------------------------------------------------------------------------

\section{Results}


\begin{figure}[hbtp]
\includegraphics[scale=0.7]{../graphics/meanEnergyAndVariance.pdf}
\caption{}
%\label{fig:}
\end{figure}

\section{Conclusion}

\section{Appendix}

\subsection{First trial wavefunction local energy}

using $\nabla_i^2 = \pdv{x_i}^2 + \pdv{y_i}^2 + \pdv{z_i}^2$ and that
\begin{align}
    \pdv[2]{x_1} \Psi{T_1} &= \pdv{x_1}\qty( -\alpha \omega x_1 e^{-\frac{1}{2}(r_1^2 + r_2^2)} )\nonumber\\
                                &= -\alpha\omega + (\alpha\omega x_1^2)e^{-\frac{1}{2}(r_1^2 + r_2^2)}
\end{align}
Which is the same for $y$ and $z$. Summing up each contribution and halving from the fraction in the hamiltonian and for each particle leads to
\begin{align}
    \pdv[2]{x_1} \Psi{T_1} = -3\alpha\omega + \alpha\omega (r_1^2 - r_2^2)e^{-\frac{1}{2}(r_1^2 + r_2^2)}
\end{align}
and the last $\omega^2 ( r_1^2 + r_2^2 )$ is from the positional potential element of the Hamiltonian.
\begin{align}
    E_{L_1} &= \frac{H \Psi_{T_1} \qty(\vec{r}_1, \vec{r}_2, \alpha )}{\Psi_{T_1} \qty(\vec{r}_1, \vec{r}_2, \alpha )}\nonumber \\
        &= C^{-1} e^{\frac{1}{2}\alpha \omega \qty(r_{1}^{2} + r_{2}^{2})} \frac{1}{2}\qty(-\nabla_1^2 -\nabla_2^2 + \omega^2(r_1^2 + r_2^2))
          C e^{-\frac{1}{2} \alpha \omega \qty(r_1^2 + r_2^2)} \nonumber \\
        &= e^{\frac{1}{2}\alpha \omega \qty(r_{1}^{2} + r_{2}^{2})}
            \frac{1}{2}\qty(-\alpha^2 \omega^2 (r_1^2 + r_2^2) + 6\alpha \omega + \omega^2 (r_1^2 + r_2^2))
            e^{-\frac{1}{2} \alpha \omega \qty(r_1^2 + r_2^2)} \nonumber \\
        &= \frac{1}{2}\omega^2 (r_1^2 + r_2^2)\qty(1 -\alpha^2 ) + 3\alpha \omega \label{eq:naiveLocalEnergyExpanded}
\end{align}

%\subsection{second trial wavefunction local energy}
%This is a more complicated differentiation. It is eased somewhat by the fact that
%\[ \nabla_i^2 = \pdv[2]{x_i} + \pdv[2]{y_i} + \pdv[2]{z_i} \]
%gives the same result per particle, from $r_{12} = \qty| \va{r_1} - \va{r_2} | = \sqrt{ (x_1 - x_2)^2 + (y_1 - y_2) + (z_1 - z_2)^2 }$ due to the
%double derivative negating negatives. We also see a close symmetry between $x, y, z$ in that the only difference is in the direct differentiation of
%$r_{12}$. $\pdv{r_{12i}}{q} = \frac{1}{2r_{12}}(q_1 - q_2)$. Further, we have the product rule which makes the wave equation differentiation separable into
%the form from \ref{eq:trial1} and the new exponent. Thus, beginning with $\pdv[2]{x_1}$, the rest can be extrapolated.
%\begin{align}
%    \pdv[2]{x_1} \qty( e^{\frac{1}{2}\frac{r_{12}}{1 + \beta r_{12}}})
%        &= \pdv{x_1} \qty( e^{\frac{1}{2} \frac{r_{12}}{1 + \beta r_{12}}} \vdot \frac{1}{2}\pdv{x_1}\qty( \frac{r_{12}}{1 + \beta r_{12}} ) )\nonumber \\
%    \text{using the equality } \frac{r_{12}}{1 + \beta r_{12} } = \frac{1}{\frac{1}{r_{12}} + \beta } \label{eq:equivalence} \\
%    \pdv{x_1} \qty( \frac{r_{12}}{1 + \beta r_{12}} )
%        &= -\qty( \frac{1}{r_{12}} + \beta )^{-2} \vdot \pdv{x_1} \qty( \frac{1}{r_{12}} ) \nonumber\\
%    \pdv{x_1}\qty( \frac{1}{r_{12}} ) &= -\frac{1}{r_{12}^2}(x_1 - x_2)\nonumber \\
%    \pdv{x_1} \qty( \frac{r_{12}}{1 + \beta r_{12}} )
%        &= - \frac{(x_1 - x_2)}{\qty( 1 + \beta r_{12} )^2} \qq{writing back into \ref{eq:equivalence}} \nonumber \\
%    \pdv[2]{x_1} \qty( e^{\frac{1}{2}\frac{r_{12}}{1 + \beta r_{12}}})
%        &= \frac{1}{2} \pdv{x} \qty( e^{\frac{1}{2} \frac{r_{12}}{1 + \beta r_{12}}} \vdot \qty( - \frac{(x_1 - x_2)}{\qty( 1 + \beta r_{12} )^2} ) ) \nonumber\\
%        &= \frac{1}{2} \qty( \qty(\frac{(x_1 - x_2)}{\qty( 1 + \beta r_{12} )^2}))^2 - \pdv{x_1} \qty( \frac{(x_1 - x_2)}{\qty( 1 + \beta r_{12} )^2} )
%\end{align}

\subsection{Trial energy transformation\label{App:trialEnergy}}
\begin{align}
    \ev{H} &= \frac{ \ev{H}{\Psi_T} }{ \braket{\Psi_T} } \nonumber \\
        &= \frac{
        \int \dd\vec{r} \Psi_T^* ( \vec{r} ) H ( \vec{r} ) \Psi_{T} ( \vec{r} )
        }{
            \int \dd\vec{r} \Psi_T^* ( \vec{r} ) \Psi_{T} ( \vec{r} ) } \\
        &= \frac{ \int \dd\vec{r} \Psi_T^* E_L \Psi_T }{ A }
        \qq{rewriting} H\Psi_T = E_L\Psi_T
        \qq{and} \int\dd\vec{r} \Psi_T^* \Psi_T = A \nonumber \\
        &= \frac{1}{A} \int\dd\vec{r} E_L(\vec{r}) P\qty(\vec{r})\, A \qq{using} P\qty(\vec{r})\equiv \frac{\Psi_T^*\Psi_T}{A} \nonumber\\
        &= \int\dd\vec{r} E_L\qty(\vec{r}) P\qty(\vec{r})
\end{align}

\subsection{ $\Psi_{T_1}$ with electron electron interaction }


%\begin{table}[h!tb]
%    \centering
%    \caption{$\Psi_{T_1}$ with Hamiltonian including electro-electron interaction}
%    \pgfplotstabletypeset[sci]{../data/trial1Full/trialwf1fullomega1.dat}
%\end{table}
\begin{table}[h!tb]
    \centering
    \caption{Table of first run}
    \pgfplotstabletypeset[
        sci,
        columns/alpha/.style={column name={$\alpha$},
        columns/energy/.style={column name={$\langle E \rangle$},
        columns/variance/.style={column name={$\sigma^2$},
        columns/distance/.style={column name={$|r_{12}|$},
        columns/accept/.style={column name={acceptance}}
    ]{../data/TwoParticlesNonInteractiveWF.dat}
\end{table}


\bibliography{\string~/Documents/bibliography/Bibliography}
\end{document}
