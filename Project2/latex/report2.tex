\documentclass[10pt, twocolumn]{article}
\usepackage{amsmath}
\usepackage{graphicx}
\usepackage{dcolumn}
\usepackage{bm}
\usepackage[colorlinks=true,urlcolor=blue,citecolor=blue]{hyperref}
\usepackage{color}
\usepackage[numbers]{natbib}
\usepackage{dsfont}
\usepackage{algorithm2e}
\usepackage{algpseudocode}

%\begin{figure}[hbtp]
%\includegraphics[scale=0.4]{filename.extension}
%\caption{description of figure} 
%\label{honestly, I don't know what label does}
%\end{figure}

\begin{document}
\title{Project 1}
\author{Anders Eriksen}

\maketitle

\begin{abstract}
%"Accurate and informative" (5pts)
    Approaching boundary value problems and extracting resulting eigenvalues from scaled 
    equations to different systems. Reinterpret the scaled equations through
    discretization and rewriting them as matrix operations on vectors. Finally, through
    similarity transformations, the eigenpairs of the system is extracted. To build
    intuition, the Jacobi method is used to approximate a diagonal matrix through 
    rotations of matrix elements.

\end{abstract}

\section{Introduction}
%"status of problem, major objectives" (10pts)
The main motivation of the endeavour is to understand and deconstruct eigenvalue problems through
boundary values systems as well as furthering understanding of similarity transformations to extract
eigenpairs from scaled equations. Finally, to make use of unit testing to ensure mathematical 
precision and to clarify code.\\

I begin by examining a system with a buckling beam, whose energetic values are extracted through 
scaling the basic differential equation and taylor approximating. Next, creating a tridiagonal 
Toeplitz matrix of the discretized approximation. After confirming the mathematical precision of my 
methods, I look towards a more complex equation for an electron in a sphreically symmetric harmonic
oscillator potential.\\
The equation is scaled and we return very close to the scaled equation of the
buckling beam. With an added piece to the current position's effect on the outcome. We discretize and 
create a new tridiagonal matrix, with an additional variable along the diagonal elements.\\
Finally, we move on to a system of 2 electrons caught in a potential. A so-called quantum dot. This we 
also scale, discretize and form a tridiagonal matrix around. Again we find the only real change is another added
variable along the diagonal. 

\section{Theory, algorithms and methods}
%"Discussion of methods used and their basis/suitability" (20pts)
2-point boundary value problems are applicable to numerous systems and are notable in that they leaed to
discrete eigenvalues. As such they are ideal for solving eigenvalue problems for, e.g.\ energies.
Our initial methods model is a buckling beam between 2 fastened points. This has analytical solutions 
and is therefore a good start to test our methods. We begin with the equation\cite{project2pdf}:
\[
\gamma \frac{d^2 u(x)}{dx^2} = -F u(x),
\]
with $\gamma$ as a system constant, the force F working on the beam at it's endpoint, a distance L 
from the origin. $u(x)$ is the displacement of the beam in the y direction. We impose the 
Dirichlet boundary conditions and set the exremes $u(0) = u(L) = 0$, our position 
$x \in \{ 0, L \}$. As an example to explore solutions to boundary value problems, we naturally 
treat the values $\gamma$, $F$ and $L$ as known quantities. \\

Our first step in our towards a solution, is to scale the function. And the first toe of this 
footstep is to introduce a dimensionless varialle $ \rho = \frac{r}{L} $, which necessarily follows 
$\rho \in [0,1]$. Next comes introducing $\lambda = \frac{FL^2}{R}$. This leads to the eigenvalue 
problem
\[
\frac{d^2u(\rho )}{d\rho^2} = -\lambda u(\rho)
\]

This can be approximated through a Tallor expansion and inserting the '-' sign from in front of our 
$\lambda$:
\[
u'' = \frac{-u(\rho + h) - u(\rho - h) + 2u(\rho)}{h^2} + {O}(h^2)
\]
with h as our stepsize.

Intending to walk from $\rho = 0$ to $1$ in $N$ steps,we derive our steplength as 
$h = \frac{\rho_N - \rho_0}{N} = \frac{1}{N}$. Letting $i \in \{ 1, 2, \ldots, N\}$ for 
\[\rho_i = \rho_0 + ih = ih\] and
\[ -\frac{u_{i+1} + u_{i-1} - 2u_{i}}{h^2} = \lambda u_i\]
This relationship can be conveniently converted into a matrix format, which is handy in linalg 
operations to extract sollutions. 
\begin{equation}
    \begin{bmatrix} d& a & 0   & 0    & \cdots  &0     & 0 \\
                                a & d & a & 0    & \cdots  &0     &0 \\
                                0   & a & d & a  &0       &\cdots & 0\\
                                \dots  & \cdots & \cdots & \cdots  &\cdots      &\cdots & \cdots\\
                                0   & \cdots & \cdots & \cdots  &a  &d & a\\
                                0   & \cdots & \cdots & \cdots  &\cdots       &a & d\end{bmatrix} 
                                 \begin{bmatrix} u_1 \\ u_2 \\ u_3 \\ \dots \\ u_{N-2} \\ u_{N-1}\end{bmatrix} 
                                     = \lambda \begin{bmatrix} u_1 \\ u_2 \\ u_3 \\ \dots \\ u_{N-2} \\ u_{N-1}\end{bmatrix}
\label{eq:matrixse} 
\end{equation}
Our tridiagonal matrix has elements $d$ along the diagonal and a along the 
off diagonals. We set $d = \frac{2}{h^2}$ and $a = -\frac{1}{h^2}$. We also keep the $0^{th}$ 
and $N^{th}$ elements out of the matrix, in part as our endpoints are known, but also because 
they are set. They will never change, because they are fastened. In my code, I refer to these 
elements as \textit{diagonal} and \textit{semidiagonal} to clarify. 

This setup has analytical eigenvalues. They follow the form
\[ \lambda_j = d + 2acos(\frac{j\pi}{N+1}) \;\; j = 1, 2, \ldots, N-1 \]\\


There is a solution to these matrices, where a matrix A can be transformed through a similarity 
transformation so that $\hat{S}^T \hat{A} \hat{S} = \hat{B}$ where $\hat{B}$ is a diagonal matrix 
with eigenvalues $\lambda$ along the diagonal. A similarity transformation also requires that 
$\hat{S}^T\hat{S} = \mathds{1}$, or $\hat{S}$ is unitary. 

First off, we must show that the unitatry transformations retain orthogonality. 
We start off with an assumed orthonormal set $v_i$, so
\[
\vec{v_i}^T \cdot \vec{v_j} = \delta_{ij}
\]
A unitary transformation means an operator $U$ working on $v_i$.
With a $\vec{w_{i}} = \hat{U}\vec{v_{i}}$, for which 
$\vec{w_{i}}^T = \vec{v_{i}}^T\hat{U}^T$ we have 
\[
    \vec{w_{i}}^T\vec{w_{j}} = \vec{v_i} \hat{U}^T \hat{U} \vec{v_j} 
\]
\[
    \hat{U}^T \hat{U} = \mathds{1} 
\]
\[
    \vec{w_i}^T\vec{w_j} = \vec{v_i}^T\vec{v_j} = \delta_{i,j}
\]\\
In other words, as long as our transformations are unitary, the vectors retain 
orthogonality. They are still orthogonal, but not the same. \\

Now, $1$ similarity transformation is not necessarily enough, but several repeated such can 
bring us where we want to go. This is where Jacobis's method comes into play. The matrix we use is
an identity matrix, where $4$ elements have been changed. $a_{ll} = a_{kk} = \cos(\theta)$, 
$a_{kl} = -a_{lk} = -\sin(\theta)$, $j \neq k$. One similarity transformation leads to a change in 
the following elements: 
\[ b_{ii} = a_{ii} \; \; i \neq k, i \neq l \]
\[ b_{ik} = a_{ik}\cos(\theta) - a_{il}\sin(\theta) \; \; i \neq k, i \neq l \]
\[ b_{il} = a_{il}\cos(\theta) + a_{ik}\sin(\theta) \; \; i \neq k, i \neq l \]
\[ b_{kk} = a_{kk}\cos^2(\theta) - 2a_{kl}\cos(\theta)\sin(\theta) + a_{ll}\sin^2(\theta) \]
\[ b_{ll} = a_{ll}\cos^2(\theta) + 2a_{kl}\cos(\theta)\sin(\theta) + a_{kk}\sin^2(\theta) \]
\[ b_{kl} = (a_{kk} - a_{ll})\cos(\theta)\sin(\theta) + a_{kl}(\cos^2(\theta) - \sin^2(\theta)) \]

For each such rotation, we choose our angles so that $a_{kl} = 0$. This is repeated until each non-diagonal
matrix element becomes 0, or at least within a tolerance of 0. Note that the elements $b_{il} \& b_{ik}$
change with each rotation as well. This means that for each rotation we disturb slightly a few other 
elements. When transforming a matrix as such, with each rotation, we find the largest element and designate
this as our $a_{kl}$. Eventually, as the elements perturbed by each transformation are scaled by numbers less
than 1, we must eventually reach a point where the only remainder is the diagonlal. At which point, we have
our $\hat{B}$ with eigenvalues. With each step, the sum of all off-diagonal elements in $\hat{B}$ be given 
as $off{(\hat{B})}^2 = off{(\hat{A})}^2 - 2a_{kl}$ and we see that per iteration, the sum of off daigonal elements 
decreases. 

If we want $b_{kl}$ to be null, we can rearrange the expression into one for $\cot(2x)$\cite{rottmann2006matematisk}, 
which gives us $\cot(2\theta) = \tau = \frac{a_{ll} - a_{kl}}{2a_{kl}}$. We also make similar approximations to 
$\tan$, $\cos$ and $\sin$, as $t, c, s$ respectively. With $t = \frac{s}{c} $. The expression for $\cot(2\theta)$
can be rewritten as $\cot(2\theta) = \frac{1}{2}(\cot(\theta) - \tan(\theta))$ which can be solved for $t$ as 
\[ t^2 + 2\tau t - 1 = 0 \]
\[ t = - \tau \pm \sqrt{1 + \tau^2} \]
And from this, we get $c = \frac{1}{\sqrt{1 + t^2}} \& s = tc$

Our algorithm then must choose the largest nondiagonal matrix element and set this as our $a_{kl}$ before rotating the 
matrix so that this element is 0, finding here the values for $s$ and $c$. After inserting the values into the 
relevant elements, we go back to the top. This repeats as long as the largest offdiagonal is greater than a tolerance 
for 0.

\begin{algorithm}
    \While{ $ \max(| a_{ij}| ) > \epsilon $ }{
        $p, q = indices(\max(|a_{ij}))$
        \eIf{$ a_{kl} \neq 0.0$ }{
            \[\tau = \frac{a_{qq} - a_{pp} }{2\, a_{pq} }\]
            \eIf{ $\tau \geq 0$ }{
                \[t = \frac{1.0}{ \tau + \sqrt{ 1 + \tau^2 } }\]
            }{
                \[t = -\frac{1.0}{ - \tau + \sqrt{ 1 + \tau^2 } }\]
            }
            \[c = \frac{1.0}{ 1.0 t^2}\]
            \[s = c\, t\]
        }{
            \[c = 1.0\]
            \[s = 0.0\]
        }
        \[ a\_ pp = a_{pp} \]
        \[ a \_ qq = a_{qq} \]
        \[ a_{pp} = c^2\, a\_ pp - 2.0\, c\,s\, a_{kl} + s^2\, a\_ qq \]
        \[ a_{qq} = s^2\, a\_ pp + 2.0\, c\, s\, a_{kl} + c^2\, a\_ qq \]
        \[ a_{kl} = 0.0 \]
        \[ a_{lk} = 0.0 \]
        \For{ $i=0$ \KwTo{} $n-1$ }{
            \uIf{ $i \neq p$ \bf{and} $i \neq q$ }{
                \[ a\_ ip = a_{ip} \]
                \[ a\_ iq = a_{iq} \]
                \[ a_{ip} = c\, a\_ ip - s\, a\_ iq \]
                \[ a_{ip} = c\, a\_ iq + s\, a\_ ip \]
                \[ a_{pi} = a_{ip} \]
                \[ a_{qi} = a_{iq} \]
            }
            \[ r\_ ip = r_{ip} \]
            \[ r\_ iq = r_{iq} \]
            \[ a_{ip} = c\, r\_ ip - s\, r\_ iq \]
            \[ a_{iq} = c\, r\_ iq + s\, r\_ ip \]
        }
        iterations ++
    }
    \caption{Algorithm to perform a Jacobi transformation on a tridiagonal matrix A, as well as with 
        a matrix R to store the eigenvectors}
\end{algorithm}

%insert algo for tridiagonlization without HO potential, with HO potential and for 2 particles.
\begin{algorithm}
    \[ a_{00} = d + potential{(0)} \]
    \[ a_{{(N-1)}{(N-1)}} = d + potential{(N-1)} \]
    \[ a_{01} = a = a_{{(N-1)}{(N-2)}} \]
    \For{ $i = 1$ \KwTo{} $i < {(N-1)}$}{
        \[ a_{ii} = d + potential{(i)} \]
        \[ a_{i{(i+1)}} = a  = a_{i{(i-1)}} \]
    }
    \caption{Algortihm for creating a tridiagonal Toeplitz matrix from an $n\times n$ dimensional matrix. 
        The potential here refers to either nothing, as in the case of the buckling beam, $\rho^2$ as in the
        single particle harmonic oscillator and also $\omega_r^2\, \rho^2 + \frac{1}{\rho}$ for the case with 
        2 particles.}
\end{algorithm}

Along the way here, it is necesary to ensure that our code follows the mathematical rules 
established prior. To do this, we need to implement unit tests. Good ones could be ensuring 
that orthogonality is preserved, that we actually find our eigen values and that we do in fact
pick out the larges off diagonal element in each step. A last test, is that for a $2\times 2$ matrix, 
there should only ever be 1 rotation necessary for jacob's method. 

Once we have written our algorithm for extracting the eigen pairs, we can set our sights to 
more interesting problems, such as electrons in spherically symetrical harmonic oscillator 
potentials. To begin, we need to incorporate the radial solution of the Schroedinger equation

\begin{equation*}
  -\frac{\hbar^2}{2 m} \left ( \frac{1}{r^2} \frac{d}{dr} r^2
  \frac{d}{dr} - \frac{l (l + 1)}{r^2} \right )R(r) 
     + V(r) R(r) = E R(r).
\end{equation*}

We introduce dimensionless variables and scale the equations until we get 

\begin{equation*}
  -\frac{d^2}{d\rho^2} u(\rho) + \rho^2u(\rho)  = \lambda u(\rho) 
\end{equation*}

Which means we essentially only need to add a term $\rho^2$ to the diagonal element of our 
matrix, and we can find the energies and energy eigentsates of the system. 

Further on, we can model a 2-particle system, a so called quantum dot\cite{project2pdf},
where we now need to scale a sum of the 2 electron quantum states. 

After some hullaballoo, we get to the expression 
\begin{equation*}
  -\frac{d^2}{d\rho^2} \psi(\rho) + \omega_r^2\rho^2\psi(\rho) +\frac{1}{\rho} = \lambda \psi(\rho)
\end{equation*}

where  $\psi(\rho)$ is the state, and analogous to our $u(\rho)$ previously. Which, if we stare at it for a little
while, means that we need to add $\omega_r^2\rho^2+1/\rho$ to our diagonal insteafd of merely $\rho^2$. 
Otherwise, the system is the same. 

\section{Results and discussion}
%"Analysis of the results and effectiveness of selection and presentation. Results well 
% understood and discussed" (20pts)
The recorded rotations for the $1^{st}$ iteration of my program were deleted, so I have no direct records 
of the runtimes and the earliest ones contain some blatant errors, where all elements were set to infinity or 
$NaN$. I do have records from runs throughout the iterations. In particular from when I was testing for number
of steps necessary to get within 4 leading digits of the eigenvalues. for 300 steps, I recorded $156093 \sim 1.73N^2$
rotations. for 400 steps, I got $278970 \sim 1.74N^2$ rotations. For 500 steps, I got $437584 \sim 1.75N^2$. These 
numbers are slowly increasing, but not near the average number of rotations suggested in the project 
literature\cite{hjorth2000computational} of 3 to 5 $N^2$. These rotations took about  275 seconds from 300 steps, to 
3500 seconds at 500 steps. This is, in other words a rather demanding task, Though I cannot say wxactly why this is, 
I doubt I have stumbled upon a hidden efficiency in the algorithm and rather some error I did not pick up in my tests. 

The tests I did run, checked the likes of preservation of orthogonality, checking that the scalar product of 2 collumns 
returned 0 for any $i \neq j$. I also made sure that I successfully captured the largest off-diagonal element in each 
run, so that the while loop ran accordingly. Finally we check that the eigenvalues we find correspond to the analytical 
ones. And another one to ensure that it needs 1 rotation for a $2\times 2$ matrix. This last one returns 2, but that is '
mainly because of the way I set my maxoffdiag in the beginning. A better solution would probably be to run 1  rotation 
and then check that the result is as it should be. 

These tests are calibrated for the system of the buckling beam, and sadly I cold not find a good way to split it. 
Therefore the tests of eigenvalues is not going to return the proper values.\\

Moving onto the QM system of a particle in a harmonic oscillator potential. The main difference in the program is a change along the diagonal. 
This means that the eigenvalues will change, though one can find them analytically as 3, 5, 11, 15, ..\cite{project2pdf}
We have to make sure that our eigenvalues still correspond to the analytical ones, and this is also a test of the program on heavier matrices. 
To get within 4 leading digits of the result, the steps have to go towards 500. This makes the jacobi rotations themselves run for an hour, 
as seen previously.

One thing that should have been tested for, but which unfortunately was not, is that the system is stable for choices of $\rho$ and $N$, 
where the lowest energies should nonetheless lie at $~3$ throughout. The strange behaviour of figure 1 points to this. 

\begin{figure}[hbtp]
\includegraphics[scale=0.4]{ProbvsDistanceFreq.png}
\caption{The probabilities of finding the electrons in a state $\psi$ with relative distance $\rho$ between eachother.
        There does seem to be a slight issue, in that the greater frequencies and thus also the greater potentials 
        seem to have a preference towards greater distance. A steeper curve on the potential should lead to a squeeze 
        towards the center of the potential and thus also less distance between them.} 
\label{honestly, I don't know what label does}
\end{figure}

\section{Conclusions}
%"conclusions, Discussion and critical comments on what was learned about the method used 
% and results obtained. Possible directions for future improvement?" (10pts)
The jacobi method is a direct method. It takes the whole matrix and operates on it untill it is in the shape we want. 
This also means that for more comlplex problemts, the FLOPS increase greatly. As we can see with the amount of time it took to get any 
particular precision with the single particle HO potential system, the method has sombottle necks.  Especially since we know our 
matrices are tridagonal, which have convenient properties that can be utilized by other methods, e.g. guaranteed convergence.\\

As stated in figure 1, the probabilities do not seem to comply with the intuition of the system, and there is, if anything a greater relative
distance between the electrons with a greater oscillator frequency. There seem to be an inversion of sorts of the attractive force on the
electrons. While the direction of the integration seems to be off, I did ensure to test the general acccuracy of the approximation to 0
for the off-diagonal elemetns. The average sum of all non-diagonal elements at this stage is around the $10^{-10}$ and so at the very least,
the eigenvalues extracted through the method should hold. the fault probably then lies somewhere in the tridiagonalization of the matrix. \\

As a final thought, I personally feel the program lost a lot of oversigth as the project went on, but had trouble making it clearer. I think segmenting 
the main and the remaining funcitons would go a long way in cleaning this up. Furthermore, I would have liked to send in a function call to 
my toeplitz tridiagonal function, which would tell it which potentials to add to the diagonal elements. This would have made the unit 
tests maintain clarity and improved the readability of my code. As a personal note, I think a more comprehensive journal as well as a 
closer planning of the code beforehand would have helped a lot in expediting the code. Secondly, I should have focused more on simply 
finishing a working example of the code early to prepare for implementing the various changes. Overall, I should have made more 
schematics of my workflow to highlight important parts of the project early on.

\bibliography{mybib}
\bibliographystyle{plain}
\end{document}
%%%%

% Code, implementation and tests
%"Readability of code, implementation, testing and discussion of benchmarks" (20)

%"clarity of figures and general presentation. too much vs too little." (10pts)

%"referencing: relevant works cited accurately?" (5pts)
