\documentclass[10pt, twocolumn]{article}
\usepackage{amsmath}
\usepackage{graphicx}
\usepackage{dcolumn}
\usepackage{bm}
\usepackage[colorlinks=true,urlcolor=blue,citecolor=blue]{hyperref}
\usepackage{color}
\usepackage[numbers]{natbib}
\usepackage{dsfont}

%\begin{figure}[hbtp]
%\includegraphics[scale=0.4]{filename.extension}
%\caption{description of figure} 
%\label{honestly, I don't know what label does}
%\end{figure}

\begin{document}
\title{Project 1}
\author{Anders Eriksen}

\maketitle

\begin{abstract}
%"Accurate and informative" (5pts)

\end{abstract}

\nocite{*}

\section{Introduction}
%"status of problem, major objectives" (10pts)

\section{Theory, algorithms and methods}
%"Discussion of methods used and their basis/suitability" (20pts)
We examine  2-point boundary value systems.this leads to discrete energies, in effect eigenvalues. 
Our initial methods model a buckling beam between 2 fastened points. This has analytical solutions 
and is therefore a good start to test our methods. Our starting point is:
\[
\gamma \frac{d^2 u(x)}{dx^2} = -F u(x),
\]
with $\gamma$ as a system constant, the force F working on the beam at it's endpoint, a distance L 
from the origin, and $u(x)$ is the displacement of the beam in the y direction. We impose the 
Dirichlet boundary conditions and set the exremes $u(0) = u(L) = 0$. Naturally, our position 
$x \in \{ 0, L \}$. As an example to explore solutions to boundary value problems, we naturally 
treat the values $\gamma$, $F$ and $L$ as known quantities. \\

Our first step in our towards a solution, is to scale the function. And the first toe of this 
footstep is to introduce a dimensionless variable $ \rho = \frac{r}{L} $, which necessarily follows 
$\rho \in [0,1]$. Next comes introducing $\lambda = \frac{FL^2}{R}$. This leads to the eigenvalue 
problem
\[
\frac{d^2u(\rho )}{d\rho^2} = -\lambda u(\rho)
\]

This can be approximated through a Taylor expansion and inserting the '-' sign from in front of our 
$\lambda$:
\[
u'' = \frac{-u(\rho + h) - u(\rho - h) + 2u(\rho)}{h^2} + {O}(h^2)
\]
With h as our stepsize.\

intending to walk from $\rho = 0$ to $1$ in $N$ steps, we naturally derive our steplength as 
$h = \frac{\rho_N - \rho_0}{N} = \frac{1}{N}$. Letting $i \in \{ 1, 2, \ldots, N\}$ for 
\[\rho_i = \rho_0 + ih = ih\] and
\[ -\frac{u_{i+1} + u_{i-1} - 2u_{i}}{h^2} = \lambda u_i\]
This relationship can be conveniently converted into a matrix format, which is handy in linalg 
operations to extract sollutions. 
\begin{equation}
    \begin{bmatrix} d& a & 0   & 0    & \cdots  &0     & 0 \\
                                a & d & a & 0    & \cdots  &0     &0 \\
                                0   & a & d & a  &0       &\cdots & 0\\
                                \dots  & \cdots & \cdots & \cdots  &\cdots      &\cdots & \cdots\\
                                0   & \cdots & \cdots & \cdots  &a  &d & a\\
                                0   & \cdots & \cdots & \cdots  &\cdots       &a & d\end{bmatrix} 
                                 \begin{bmatrix} u_1 \\ u_2 \\ u_3 \\ \dots \\ u_{N-2} \\ u_{N-1}\end{bmatrix} 
                                     = \lambda \begin{bmatrix} u_1 \\ u_2 \\ u_3 \\ \dots \\ u_{N-2} \\ u_{N-1}\end{bmatrix} . 
\label{eq:matrixse} 
\end{equation}
Where our tridiagonal matrix has elements "d" along the diagonal, naturally, and a along the 
off diagonals. I don't fully see the naming convention, but it makes linking the research 
through the course resources far easier. In my code, I refer to these elements as "diagonal" 
and "semidiagonal" to clarify. We set $d = \frac{2}{h^2}$ and $a = -\frac{1}{h^2}$. We also keep 
the $0^{th}$ and $N^{th}$ elements out of the matrix, in part as our endpoints are known, but 
also because they are set. They will never change, because they are fastened. 

The method we wish to use here, is the Jacobi method. It makes use of unitary transformations to 
rotate the matrix until we've reduced the rows of our tridiagonal matrix to a full diagonal matrix. 
The numbers remaining on the diagonal is then our eigenvectors. In addition, we maintain the 
rotations on an identity matrix, which then takes in our eigenvectors corresponding to the 
eigenvalues in the original matrix. 

first off, we must show that the unitatry transformations retain orthogonality. 

We start off with an assumed orthonormal set $v_i$, so
\[
\vec{v_i}^T \cdot \vec{v_j} = \delta_{ij}
\]
A unitary transformation means an operator $U$ working on $v_i$.
With a $\vec{w_{i}} = \hat{U}\vec{v_{i}}$, for which 
$\vec{w_{i}}^T = \vec{v_{i}}^T\hat{U}^T$ we have 
\[
    \vec{w_{i}}^T\vec{w_{j}} = \vec{v_i} \hat{U}^T \hat{U} \vec{v_j} 
\]
\[
    \hat{U}^T \hat{U} = \mathds{1} 
\]
\[
    \vec{w_i}^T\vec{w_j} = \vec{v_i}^T\vec{v_j} = \delta_{i,j}
\]
    
\section{Code, implementation and tests}
%"Readability of code, implementation, testing and discussion of benchmarks" (20)

\section{Analysis}
%"Analysis of the results and effectiveness of selection and presentation. Results well understood and discussed" (20pts)


\section{conclusions}
%"conclusions, Discussion and critical comments on what was learned about the method used and results obtained. Possible directions for future improvement? (10pts)


\bibliography{mybib}
\bibliographystyle{plain}
\end{document}
%%%%
%"clarity of figures and general presentation. too much vs too little. (10pts)

%"referencing: relevant works cited accurately? (5pts)
