\documentclass[10pt, twocolumn]{revtex4-1}
\usepackage{amsmath}
\usepackage{graphicx}
\usepackage{dcolumn}
\usepackage{bm}
\usepackage[colorlinks=true,urlcolor=blue,citecolor=blue]{hyperref}
\usepackage{color}

%\begin{figure}[hbtp]
%\includegraphics[scale=0.4]{filename.extension}
%\caption{description of figure} 
%\label{honestly, I don't know what label does}
%\end{figure}

\begin{document}
\title{Project 1}
\author{Anders Eriksen}
\begin{abstract}
%"Accurate and informative" (5pts)

\end{abstract}
\maketitle

\section{Introduction}
%"status of problem, major objectives" (10pts)

\section{Theory, algorithms and methods}
%"Discussion of methods used and their basis/suitability" (20pts)
We examine  2-point boundary value systems. this leads to discrete energies, in effect eigenvalues. Our initial methods model a buckling beam between 2 fastened points. This has analytical solutions and is therefore a good start to test our methods. Our starting point is:
\[
\gamma \frac{d^2 u(x)}{dx^2} = -F u(x),
\]
with $\gamma$ as a system constant, the force F working on the beam at it's endpoint, a distance L from the origin, and $u(x)$ is the displacement of the beam in the y direction. We do impose the Dirichlet boundary conditions and set the exremes $u(0) = u(L) = 0$. 

\section{Code, implementation and tests}
%"Readability of code, implementation, testing and discussion of benchmarks" (20)

\section{Analysis}
%"Analysis of the results and effectiveness of selection and presentation. Results well understood and discussed" (20pts)


\section{conclusions}
%"conclusions, Discussion and critical comments on what was learned about the method used and results obtained. Possible directions for future improvement? (10pts)

\end{document}
%%%%
%"clarity of figures and general presentation. too much vs too little. (10pts)

%"referencing: relevant works cited accurately? (5pts)
