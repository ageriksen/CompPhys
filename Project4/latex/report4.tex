\documentclass[10pt, twocolumn]{revtex4-1}
\listfiles               %  print all files needed to compile this document

\usepackage{amsmath}
\usepackage{xparse}
\usepackage{graphicx}
\usepackage{dcolumn}
\usepackage{bm}
\usepackage[colorlinks=true,urlcolor=blue,citecolor=blue]{hyperref}
\usepackage{color}
\usepackage{physics}
\usepackage{algorithm2e}
\usepackage{algpseudocode}
\usepackage{pgfplots}
\usepackage{natbib}

\pgfplotsset{compat=1.15}

%\begin{figure}[hbtp]
%\includegraphics[scale=0.4]{test1.pdf}
%\caption{Exact and numerial solutions for $n=10$ mesh points.} 
%\label{fig:n10points}
%\end{figure}

%\begin{tikzpicture}
%    \begin{axis}[
%            title= Earth-Sun system, Forward Euler integration,
%            xlabel={$x$},
%            ylabel={$y$},
%        ]
%        \addplot table {../runresults/earthEuler2body.dat}
%    \end{axis}
%\end{tikzpicture}

\begin{document}

\title{%
    Project 4\\
    \large Studies of phase transition in magnetic systems}
\author{Anders Eriksen}
\begin{abstract}
\end{abstract}
\maketitle

\section{Introduction}


\section{Methods and theory}

We want to study a 2 dimensional ferromagnetic system through the Ising model, specifically in phase transitions.
The system we're studying has an energy 
\[ E = -J \sum_{<kl>}^N s_k \, s_l \]
In which the "$<kl>$" signifies summing over neighbouring spots in the lattice only. $s_k = \pm 1$, and N is the total number of spins in 
the latice. J is a coupling constant and, as we are currently investigating ferromagnetic elements we have $J > 0$.
As we will investigate changes in the system, the equation can be solved as
\[ \Delta E = 2 Js_l^1 \sum_{<k>}^N s_k \].

The numerical issues we focus on through the project, is periodic boundary conditions as well as the metropolis algorithm. % Look further into 
%these and discuss here. 

As with most systems, we begin with a simple iteration. For the ising model of ferromagnets, we start out with a $2\times 2$ lattice to find $\ev{E}$
as well as the mean average magnetic moment, $\abs{M}$, henceforth "mean magnetism". Additional valuable properties are the specific heat capacity, $C_V$
as well as succeptability. $\chi$ as functions of T, with periodic boundary conditions. 

Having studied simple system, we will write a code of the Ising model to analyse it and use our previous results as benchmarks for further expansion. We 
want to accurately extract the properties found previously, for a scaled temperature $T = 1 \qq{with dimensions} \qty[ T ] = \frac{ J }{ kT }$. Here, 
k is the Boltzman constant. This will allow us to cancel out the k in future calculations. Another important step, is to log the number of Monte Carlo
(MC) cycles we run. 

Having Benchmarked and controlled our program for a simple system, we up the lattice size $L\times L$ to $L=20$. Because there is little value in studying 
a system not in equilibrium for our purposes and in order to not waste cpu cycles on usesless info that will at best be thrown out, we need to determine the
time the system needs to reach equilibrium. The simplest method here, is to just plot the various expectation values as functions of the number of MC cycles.
and determining by-eye when the system reaches the desired stability. We measure time in sweeps per lattice of MC cycles. An interesting question, is whether
or not the equilibration time depends on the starting position. Whether this is ordered so all spins point in the same direction or random configuration.
These first tests will be run with the initial temperature $T=1$. We will also investigate the temperature $T = 2.4$ and whether or not it is possible to 
estimate equilibration time. Here, we want to plot the total number of accepted MC cycles as a function of the total number of MC cycles run, so as to 
investigate whether or not we can attest to the accepted configurations depending on T. 

Additionally we want to 


\section{Results and discussion}

\section{conclusion}

\section{appendix}


\bibliography{\string~/Documents/bibliography/Bibliography}
\end{document}
