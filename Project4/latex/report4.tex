\documentclass[10pt, twocolumn]{revtex4-1}
\listfiles               %  print all files needed to compile this document

\usepackage{amsmath}
\usepackage{xparse}
\usepackage{graphicx}
\usepackage{dcolumn}
\usepackage{bm}
\usepackage[colorlinks=true,urlcolor=blue,citecolor=blue]{hyperref}
\usepackage{color}
\usepackage{physics}
\usepackage{algorithm2e}
\usepackage{algpseudocode}
\usepackage{pgfplots}
\usepackage{natbib}

\pgfplotsset{compat=1.15}

%\begin{figure}[hbtp]
%\includegraphics[scale=0.4]{test1.pdf}
%\caption{Exact and numerial solutions for $n=10$ mesh points.} 
%\label{fig:n10points}
%\end{figure}

%\begin{tikzpicture}
%    \begin{axis}[
%            title= Earth-Sun system, Forward Euler integration,
%            xlabel={$x$},
%            ylabel={$y$},
%        ]
%        \addplot table {../runresults/earthEuler2body.dat}
%    \end{axis}
%\end{tikzpicture}

\begin{document}

\title{%
    Project 4\\
    \large Studies of phase transition in magnetic systems}
\author{Anders Eriksen}
\begin{abstract}
\end{abstract}
\maketitle

\section{Introduction}


\section{Methods and theory}

We want to study a 2 dimensional ferromagnetic system through the Ising model, specifically in phase transitions.
The system we're studying has an energy 
\[ E = -J \sum_{<kl>}^N s_k \, s_l \]
In which the "$<kl>$" signifies summing over neighbouring spots in the lattice only. $s_k = \pm 1$, and N is the total number of spins in 
the latice. J is a coupling constant and, as we are currently investigating ferromagnetic elements we have $J > 0$.
As we will investigate changes in the system, the equation can be solved as
\[ \Delta E = 2 Js_l^1 \sum_{<k>}^N s_k \].

These 


\section{Results and discussion}

\section{conclusion}

\section{appendix}


\bibliography{\string~/Documents/bibliography/Bibliography}
\end{document}
