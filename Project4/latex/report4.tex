\documentclass[10pt, twocolumn]{revtex4-1}
\listfiles               %  print all files needed to compile this document

\usepackage{amsmath}
\usepackage{xparse}
\usepackage{graphicx}
\usepackage{dcolumn}
\usepackage{bm}
\usepackage[colorlinks=true,urlcolor=blue,citecolor=blue]{hyperref}
\usepackage{color}
\usepackage{physics}
\usepackage{algorithm2e}
\usepackage{algpseudocode}
\usepackage{pgfplots}
\usepackage{natbib}

\pgfplotsset{compat=1.15}

%\begin{figure}[hbtp]
%\includegraphics[scale=0.4]{test1.pdf}
%\caption{Exact and numerial solutions for $n=10$ mesh points.}
%\label{fig:n10points}
%\end{figure}

%\begin{tikzpicture}
%    \begin{axis}[
%            title= Earth-Sun system, Forward Euler integration,
%            xlabel={$x$},
%            ylabel={$y$},
%        ]
%        \addplot table {../runresults/earthEuler2body.dat}
%    \end{axis}
%\end{tikzpicture}

\begin{document}

\title{%
    Project 4\\
    \large Studies of phase transition in magnetic systems}
\author{Anders Eriksen}
\begin{abstract}
\end{abstract}
\maketitle

\section{Introduction}


\section{Methods and theory}

We want to study a 2 dimensional ferromagnetic system through the Ising model, specifically in phase transitions.
The system we're studying has an energy
\[ E = -J \sum_{<kl>}^N s_k \, s_l \]
In which the "$<kl>$" signifies summing over neighbouring spots in the lattice only. $s_k = \pm 1$, and N is the total number of spins in
the latice. J is a coupling constant and, as we are currently investigating ferromagnetic elements we have $J > 0$.
As we will investigate changes in the system, the equation can be solved as
\[ \Delta E = 2 Js_l^1 \sum_{<k>}^N s_k \].

The numerical issues we focus on through the project, is periodic boundary conditions as well as the metropolis algorithm. % Look further into
%these and discuss here.

As with most systems, we begin with a simple iteration. For the ising model of ferromagnets, we start out with a $2\times 2$ lattice to find
as well as the mean average magnetic moment henceforth "mean magnetism". Additional valuable properties are the specific heat capacity,
as well as succeptability as functions of T, with periodic boundary conditions.
\[
    \ev{E(T)}, \qq{} \abs{M(T)}, \qq{} C_V(T), \qq{} \chi(T)
\] % INCLUD DESCRIPTIONS OF HOW TO FIND MAGNETIZATION, HEAT CAPACITY AND SUCCEPTIBILITY!!!

Having studied simple system, we will write a code of the Ising model to analyse it and use our previous results as benchmarks for further expansion. We
want to accurately extract the properties found previously, for a scaled temperature $T = 1 \qq{with dimensions} \qty[ T ] = \frac{ J }{ kT }$. Here,
k is the Boltzman constant. This will allow us to cancel out the k in future calculations. Another important step, is to log the number of Monte Carlo
(MC) cycles we run.
%%%% algorithm begin
\begin{algorithm}
    mat<double> $Transfer = [ e^{-\Delta E/ \beta }$ ]\;
    \For{$n=1$ \KwTo{$L^2$} }{
        $x_r, y_r = RNG$\;
        $\Delta{E} = 2\cdot s\qty[x_r, y_r] \sum_{\langle{} k \rangle{} }s_k$\;
        $r = RNG_{uniform} \in{} \qty[ 0, 1 ] $\;
        \If{$r \leq Transfer(\Delta{E})$}{
            $s[x_r, y_r] *= -1.0$\;
            $M += 2\cdot s[x_r, y_r]$\;
            $E += \Delta{E}$\;
        }
    }
    \caption{}
\end{algorithm}
%%%% algorithm end

Having Benchmarked and controlled our program for a simple system, we up the lattice size $L\times L$ to $L=20$. To abvoid wasting cpu cycles on usesless
info that will at best be thrown out, we need to determine the time the system needs to reach equilibrium. The simplest method here, is to just plot the
various expectation values as functions of the number of MC cycles. and determining by-eye when the system reaches the desired stability. Time is measured in
sweeps of MC cycles per lattice. An interesting question, is whether or not the equilibration time depends on the starting position. Whether this is
ordered so all spins point in the same direction or random configuration. These first tests will be run with the initial temperature $T=1$. We will also
investigate the temperature $T = 2.4$ and whether or not it is possible to estimate equilibration time. Here, we want to plot the total number of accepted
MC cycles as a function of the total number of MC cycles run, so as to investigate whether or not we can attest to the accepted configurations depending
on T.

Additionally we want to study the probabilities for energies, $\Pr(E)$, in effect counting up the number of times the energy $E$ comes up in the computation.
We begin our counting after having reached equilibrium to ensure statistically significant readout. We want the energetic variance $\sigma_E^2$.
%discuss results!

With our system set up, we can now study the properties of the system as we change the temperature. There is a phase shift of the system on the crossing of
a certain critical temperature $T_C$. Below this, the energy is low enough, so that the bias in energetic preference freezes the system towards the energetic
minimum. This should result in growing clusters of spins, untill the system eventually inhabits the minimum energy state where all spins point in the same
direction. With temperatures greater than $T_C$, the energety present lessens the bias towards the energetically preferable state. This means that the clusters
of similar spins are smaller and more spread, with a more random distribution.

Near $T_C$ we can characterize several properties of the system with a power law behaviour. In the Ising model, the mean magnetisation is given by
\[ \ev{M(T)} \sim \qty( T - t_C )^{\beta} \qq{with} \beta = \frac{1}{8} \]
Here, $\beta$ is called the "critical exponent". % what is this?!?
Similar expressions can be found for heat capacity and susceptibility:
\begin{align*}
    C_V(T) &\sim \qty| T_C - T |^{\alpha}, \qq{} \alpha = 0 \\
    \chi(T) &\sim \qty| T_C - T |^{\gamma}, \qq{} \gamma = \frac{7}{4}
\end{align*}
The 0 exponent stems from logarithmic divergence, where the value
arises from a rewrite into the Taylor series, where $\qty| T_C - T |^{\alpha} \simeq -\ln(\qty| T_C - T | ) + O(\alpha^2)$. This is an exponentially growing
"spike" around the critical temperature.


Another important quantity is correlation length, $\xi$. For temperatures $T \gg T_C$, the correlation between the spins should be so low, that the
correlation length should be on the order of the distance between each lattice point. As we approach the critical temperature from above, this correlation
length grows with a divergent behaviour
\[ \xi(T) \sim \qty| T_C - T |^{-\nu} \]

The $2^{nd}$ order order correction is characterised by a $\xi$ spanning the system. A finite lattice, therefore has a correlation length proportional to
the lattice size. Using so-called "finite size scaling relations" we can relate the behaviour of $\xi$ to the infinite span. With this, the critical
temperature scales as
\[ T _ { C } ( L ) - T _ { C } ( L = \infty ) = a L ^ { - 1 / \nu } \]
Where $a$ is some constant and $\nu$ is as defined for $\xi$ above. When we set $T = T_C$, the above equations go to:
\begin{align*}
    \text{ mean magnetisation: } \\
    \langle \mathcal { M } ( T ) \rangle &\sim \left( T - T _ { C } \right) ^ { \beta } \rightarrow L ^ { - \beta / \nu }\\
    \text{ Heat capacity: } \\
    C _ { V } ( T ) &\sim \left| T _ { C } - T \right| ^ { - \gamma } \rightarrow L ^ { \alpha / \nu }\\
    \text{ and succeptibility: } \\
    \chi ( T ) &\sim \left| T _ { C } - T \right| ^ { - \alpha } \rightarrow L ^ { \gamma / \nu }
\end{align*}
We want to study these numerically relations in our system when it is near $T_C$. We want to examine the dependence of T and L for the listed characteristics.
We also want to time these tests, to compare between parallelization of the code. Finally, we want to compare our critical temperature above to the
closed-form solution found by Onsanger in 1944: $k T _ { C } / J = 2 / \ln ( 1 + \sqrt { 2 } ) \approx 2.269$ \cite{PhysRev.65.117}.


\section{Results and discussion}

With these values to control against, we can now benchmark our program. Writing out a few functions, we have one for setting up our lattice, one for running
through an MCcycle and one for writing the results to file. I store the expectation vales of the energy, the energy squared and the same for the magnetization,
as well as the absolute value per microstate. These can be used directly, in the case of the energy expectation value and the mean magnetization, or in
combination such as with the heat capacity and susceptibility, who's scaled values are the variance in either energy or magnetization expectation values. Both
of these values are important in any MC calculation, because they give a sharp definition of the statistical sharpness of the result.
%% insert plot of 2x2LatticeOrderUp.png and discuss.
we see that they approach the analytical values well after 1e6 cycles, with %read from outputExerciseB and reiterate the rough approach seen in the figure
%above, in order of magnitude.

\section{conclusion}

\section{appendix}


\bibliography{\string~/Documents/bibliography/Bibliography}
\end{document}
